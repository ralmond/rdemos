% Options for packages loaded elsewhere
\PassOptionsToPackage{unicode}{hyperref}
\PassOptionsToPackage{hyphens}{url}
%
\documentclass[
]{article}
\title{Confidence Interval: COVID Vaccine tests}
\author{}
\date{\vspace{-2.5em}}

\usepackage{amsmath,amssymb}
\usepackage{lmodern}
\usepackage{iftex}
\ifPDFTeX
  \usepackage[T1]{fontenc}
  \usepackage[utf8]{inputenc}
  \usepackage{textcomp} % provide euro and other symbols
\else % if luatex or xetex
  \usepackage{unicode-math}
  \defaultfontfeatures{Scale=MatchLowercase}
  \defaultfontfeatures[\rmfamily]{Ligatures=TeX,Scale=1}
\fi
% Use upquote if available, for straight quotes in verbatim environments
\IfFileExists{upquote.sty}{\usepackage{upquote}}{}
\IfFileExists{microtype.sty}{% use microtype if available
  \usepackage[]{microtype}
  \UseMicrotypeSet[protrusion]{basicmath} % disable protrusion for tt fonts
}{}
\makeatletter
\@ifundefined{KOMAClassName}{% if non-KOMA class
  \IfFileExists{parskip.sty}{%
    \usepackage{parskip}
  }{% else
    \setlength{\parindent}{0pt}
    \setlength{\parskip}{6pt plus 2pt minus 1pt}}
}{% if KOMA class
  \KOMAoptions{parskip=half}}
\makeatother
\usepackage{xcolor}
\IfFileExists{xurl.sty}{\usepackage{xurl}}{} % add URL line breaks if available
\IfFileExists{bookmark.sty}{\usepackage{bookmark}}{\usepackage{hyperref}}
\hypersetup{
  pdftitle={Confidence Interval: COVID Vaccine tests},
  hidelinks,
  pdfcreator={LaTeX via pandoc}}
\urlstyle{same} % disable monospaced font for URLs
\usepackage[margin=1in]{geometry}
\usepackage{color}
\usepackage{fancyvrb}
\newcommand{\VerbBar}{|}
\newcommand{\VERB}{\Verb[commandchars=\\\{\}]}
\DefineVerbatimEnvironment{Highlighting}{Verbatim}{commandchars=\\\{\}}
% Add ',fontsize=\small' for more characters per line
\usepackage{framed}
\definecolor{shadecolor}{RGB}{248,248,248}
\newenvironment{Shaded}{\begin{snugshade}}{\end{snugshade}}
\newcommand{\AlertTok}[1]{\textcolor[rgb]{0.94,0.16,0.16}{#1}}
\newcommand{\AnnotationTok}[1]{\textcolor[rgb]{0.56,0.35,0.01}{\textbf{\textit{#1}}}}
\newcommand{\AttributeTok}[1]{\textcolor[rgb]{0.77,0.63,0.00}{#1}}
\newcommand{\BaseNTok}[1]{\textcolor[rgb]{0.00,0.00,0.81}{#1}}
\newcommand{\BuiltInTok}[1]{#1}
\newcommand{\CharTok}[1]{\textcolor[rgb]{0.31,0.60,0.02}{#1}}
\newcommand{\CommentTok}[1]{\textcolor[rgb]{0.56,0.35,0.01}{\textit{#1}}}
\newcommand{\CommentVarTok}[1]{\textcolor[rgb]{0.56,0.35,0.01}{\textbf{\textit{#1}}}}
\newcommand{\ConstantTok}[1]{\textcolor[rgb]{0.00,0.00,0.00}{#1}}
\newcommand{\ControlFlowTok}[1]{\textcolor[rgb]{0.13,0.29,0.53}{\textbf{#1}}}
\newcommand{\DataTypeTok}[1]{\textcolor[rgb]{0.13,0.29,0.53}{#1}}
\newcommand{\DecValTok}[1]{\textcolor[rgb]{0.00,0.00,0.81}{#1}}
\newcommand{\DocumentationTok}[1]{\textcolor[rgb]{0.56,0.35,0.01}{\textbf{\textit{#1}}}}
\newcommand{\ErrorTok}[1]{\textcolor[rgb]{0.64,0.00,0.00}{\textbf{#1}}}
\newcommand{\ExtensionTok}[1]{#1}
\newcommand{\FloatTok}[1]{\textcolor[rgb]{0.00,0.00,0.81}{#1}}
\newcommand{\FunctionTok}[1]{\textcolor[rgb]{0.00,0.00,0.00}{#1}}
\newcommand{\ImportTok}[1]{#1}
\newcommand{\InformationTok}[1]{\textcolor[rgb]{0.56,0.35,0.01}{\textbf{\textit{#1}}}}
\newcommand{\KeywordTok}[1]{\textcolor[rgb]{0.13,0.29,0.53}{\textbf{#1}}}
\newcommand{\NormalTok}[1]{#1}
\newcommand{\OperatorTok}[1]{\textcolor[rgb]{0.81,0.36,0.00}{\textbf{#1}}}
\newcommand{\OtherTok}[1]{\textcolor[rgb]{0.56,0.35,0.01}{#1}}
\newcommand{\PreprocessorTok}[1]{\textcolor[rgb]{0.56,0.35,0.01}{\textit{#1}}}
\newcommand{\RegionMarkerTok}[1]{#1}
\newcommand{\SpecialCharTok}[1]{\textcolor[rgb]{0.00,0.00,0.00}{#1}}
\newcommand{\SpecialStringTok}[1]{\textcolor[rgb]{0.31,0.60,0.02}{#1}}
\newcommand{\StringTok}[1]{\textcolor[rgb]{0.31,0.60,0.02}{#1}}
\newcommand{\VariableTok}[1]{\textcolor[rgb]{0.00,0.00,0.00}{#1}}
\newcommand{\VerbatimStringTok}[1]{\textcolor[rgb]{0.31,0.60,0.02}{#1}}
\newcommand{\WarningTok}[1]{\textcolor[rgb]{0.56,0.35,0.01}{\textbf{\textit{#1}}}}
\usepackage{graphicx}
\makeatletter
\def\maxwidth{\ifdim\Gin@nat@width>\linewidth\linewidth\else\Gin@nat@width\fi}
\def\maxheight{\ifdim\Gin@nat@height>\textheight\textheight\else\Gin@nat@height\fi}
\makeatother
% Scale images if necessary, so that they will not overflow the page
% margins by default, and it is still possible to overwrite the defaults
% using explicit options in \includegraphics[width, height, ...]{}
\setkeys{Gin}{width=\maxwidth,height=\maxheight,keepaspectratio}
% Set default figure placement to htbp
\makeatletter
\def\fps@figure{htbp}
\makeatother
\setlength{\emergencystretch}{3em} % prevent overfull lines
\providecommand{\tightlist}{%
  \setlength{\itemsep}{0pt}\setlength{\parskip}{0pt}}
\setcounter{secnumdepth}{-\maxdimen} % remove section numbering
\ifLuaTeX
  \usepackage{selnolig}  % disable illegal ligatures
\fi

\begin{document}
\maketitle

Data from Moderna Vaccine Study Control Group

\begin{Shaded}
\begin{Highlighting}[]
\NormalTok{X }\OtherTok{\textless{}{-}} \DecValTok{95}
\NormalTok{N }\OtherTok{\textless{}{-}} \DecValTok{1500}
\end{Highlighting}
\end{Shaded}

In a sample of 1500 volunteers receiving the placebo, there were 95
positive cases; so our estimate for the rate of COVID-19 in this
population (and this time period) is 0.063.

Recall that the formula for the \(\alpha\) confidence interval is

C.I. \[ \bar X \pm (z_{1-\alpha/2})\ \sigma_{\bar X}\] Here
\(z_{1-\alpha/2}\) is the \(1-\alpha/2\) quantile of the normal
distribution. We can look that up on a
\href{https://pluto.coe.fsu.edu/rdemos/IntroStats/NormalCalculator.Rmd}{Normal
Table}. For \(\alpha=.95\), \(z_{1-\alpha/2}=1.96\approx 2\).

For binomial distribution \[ \bar X = X/N=\hat p\] This is the value
0.063 we calculated earlier. The \emph{hat} over the \(p\) is a sign
that it is a (maximum likelihood) estimate.

The usual formula for the standard error of a mean (from a simple random
sample) is \[ \sigma_{\bar X} = \frac{\sigma}{\sqrt{N}} \] For the
binomial distribution, the standard deviation is
\[ \sigma = \sqrt{p(1-p)}; \qquad s = \sqrt{\hat p(1-\hat p)}\] Plug
that into the formula for the standard error and we get:

\[ \sigma_{\bar X} = \sqrt{p(1-p)/N} \] Lets go ahead and calculate
those

\begin{Shaded}
\begin{Highlighting}[]
\NormalTok{p.hat }\OtherTok{\textless{}{-}}\NormalTok{ X}\SpecialCharTok{/}\NormalTok{N}
\NormalTok{se }\OtherTok{\textless{}{-}} \FunctionTok{sqrt}\NormalTok{(p.hat}\SpecialCharTok{*}\NormalTok{(}\DecValTok{1}\SpecialCharTok{{-}}\NormalTok{p.hat)}\SpecialCharTok{/}\NormalTok{N)}
\end{Highlighting}
\end{Shaded}

The probability estimate is 0.063 and the standard error is 0.0063.

I'll now use an R trick. \texttt{qnorm()} is the R function to calculate
the quantiles of the normal distribution. If I give it two
probabilities, it will give me both the postive and negative values. So
I will pass it \((\alpha/2,1-\alpha/2)\), this gives the values
\(r round(qnorm(c(.025,.975)),3)\).

Because R does calculations on vectors, it will calculate both sides of
the confidence interval with one formula.

\begin{Shaded}
\begin{Highlighting}[]
\NormalTok{ci }\OtherTok{\textless{}{-}}\NormalTok{ p.hat }\SpecialCharTok{+} \FunctionTok{qnorm}\NormalTok{(}\FunctionTok{c}\NormalTok{(.}\DecValTok{025}\NormalTok{,.}\DecValTok{975}\NormalTok{))}\SpecialCharTok{*}\NormalTok{se}
\end{Highlighting}
\end{Shaded}

Prevlance of covid \emph{at the time and in the locations the study was
run} was between (5.1\%,7.6\%).

Note that a lot of things have changed between now and then. In
particular, the rise of the much more transmissable delta variant. But
also changes in how seriously people take masking and other percautions.
In particular, there is probably considerable regional variation in the
prevalence of COVID-19.

The web site \url{https://www.microcovid.org/} tracks this on a
county-by-county basis.

\hypertarget{severe-covid}{%
\subsection{Severe Covid}\label{severe-covid}}

Same thing with the severe (hospitalizations or death) COVID numbers.

\begin{Shaded}
\begin{Highlighting}[]
\NormalTok{X1 }\OtherTok{\textless{}{-}} \DecValTok{11}
\NormalTok{p1 }\OtherTok{\textless{}{-}}\NormalTok{ X1}\SpecialCharTok{/}\NormalTok{N}
\NormalTok{se1 }\OtherTok{\textless{}{-}} \FunctionTok{sqrt}\NormalTok{(p1}\SpecialCharTok{*}\NormalTok{(}\DecValTok{1}\SpecialCharTok{{-}}\NormalTok{p1)}\SpecialCharTok{/}\NormalTok{N)}
\NormalTok{ci1 }\OtherTok{\textless{}{-}}\NormalTok{ p1 }\SpecialCharTok{+} \FunctionTok{qnorm}\NormalTok{(}\FunctionTok{c}\NormalTok{(.}\DecValTok{025}\NormalTok{,.}\DecValTok{975}\NormalTok{))}\SpecialCharTok{*}\NormalTok{se1}
\end{Highlighting}
\end{Shaded}

Prevlance of severe covid \emph{at the time and in the locations the
study was run} was between (0.3\%,1.2\%).

\end{document}
